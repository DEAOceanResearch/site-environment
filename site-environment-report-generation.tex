%% A very basic document template: 1 inch margins all round, no packages
%%
%% $Revision: 1.3 $
%% $Date: 2012/12/27 16:18:41 $
%% $Id: site-environment-report-generation.tex,v 1.3 2012/12/27 16:18:41 duncombe Exp $
%% 
\documentclass[letterpaper,12pt,oneside]{article}
\batchmode

% define the page to have 1 inch margins all round 
% \setlength{\oddsidemargin}{0pt}
% \setlength{\textwidth}{6.5in}
% \setlength{\topmargin}{0in}
% \setlength{\headsep}{0in}
% \setlength{\headheight}{0in}
% \setlength{\textheight}{9in}

\usepackage[letterpaper,left=1in,right=1.0in,top=1.0in,bottom=1in]{geometry}
\usepackage{url}


\pagestyle{empty}

\begin{document}

\title{Generate an Environment Report for a Proposed Monitoring
Site}

\author{Christopher M.\ Duncombe Rae}
\today

\abstract{When designing a mooring for a proposed monitoring
site, it is necessary and desirable to know environmental
conditions and extremes likely to be encountered during the
deployment period. A suite of software and data sources are
described, and the technique used to create such an environmental
report is explained.

\clearpage

\section{Introduction}

\section{Installation}

Use \path{cvs export -D`fdate` site-environment} to generate a clean version of
the site environment directory structure from the CVS repositiory without
generating the CVS administration files. You do not
want to do \path{cvs checkout} because you are going to be populating the
structure with data and modifying the templates for the particular site you are
working with, and inadvertently running \path{cvs ci} will save these changes back to the
repository. This is probably not what you want to do. 

/Environment Variable

export SITEPOINT=(x y)

\section{Data Sources}

\subsection{Bathymetry}

Bathymetry is obtained from GEBCO, and ETOPO-1. Sites? 



The 
multibeam bathymetry is obtained by going to
\url{http://www.ngdc.noaa.gov/mgg/bathymetry/multibeam.html}\footnote{Checked
correct
on 2012-12-20}
and clicking
on the Autochart option in the left panel which takes you to
\url{http://www.ngdc.noaa.gov/autochart/}. Enter email and select the
region, select options and submit request. An email will alert you when the
chart of multibeam bathymetry is ready to be downloaded
as a tar.gz archive. 

For the other bathymetries run the bathy 


\section{Software Used}

\subsection{Data Archive}

I store all bathymetry files in /usr/local/bathy/ in
subdirectories etopo1,	etopo2,	etopo5, tbase and	gebco.





TOPO	location of the bathymetric source netCDF files
\url{/usr/local/bathy/etopo1/ETOPO1_Ice_g_gmt4-.grd}


\subsection{Step-By-Step Procedure}

Call initialize shell script which will

\begin{itemize}
\item {
Create a directory tree 

\begin{tabular}{lll}
SITE-NAME & & \\
&  argo & \\
&  bathy & \\
&  flux & \\
&  precip & \\
&  waves & \\
&  wind & \\
\end{tabular}
}

\item populate the tree with site-environment.tex, Makefile, etc.

\item modified scripts
with specific site name and latitude and longitude.


\subsubsection{Bathymetry}

Scripts are: 




\end{document}


