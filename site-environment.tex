%% A very basic document template: 1 inch margins all round, no packages
%%
%% $Revision: 1.1 $
%% $Date: 2012/02/16 16:04:13 $
%% $Id: site-environment.tex,v 1.1 2012/02/16 16:04:13 duncombe Exp $
%% 
\documentclass[letterpaper,12pt,oneside]{article}
\batchmode

% define the page to have 1 inch margins all round 
% \setlength{\oddsidemargin}{0pt}
% \setlength{\textwidth}{6.5in}
% \setlength{\topmargin}{0in}
% \setlength{\headsep}{0in}
% \setlength{\headheight}{0in}
% \setlength{\textheight}{9in}


%%%%%%%%%%%%%%%%%%
% include packages
%%%%%%%%%%%%%%%%%%
\usepackage[letterpaper,left=1in,right=1.0in,top=1.0in,bottom=1in]{geometry}
\usepackage{url}

% some potentially useful packs
\usepackage{graphicx}
\usepackage{url}


%%%%%%%%%%%%%%%%%%%%%%
% for the bibliography 
%%%%%%%%%%%%%%%%%%%%%%
%% Use one of natbib or apalike (or another)
% \usepackage[square]{natbib} 
% \usepackage[square]{apalike} 
%% set a new line for the references section 
% \newcommand{\refheading}{Bibliography}
% \AtBeginDocument{\renewcommand\refname{\large \refheading}}
%% then define one of 
% \bibliographystyle{ametsoc}
% \bibliographystyle{/usr/share/texmf/bibtex/bst/natbib/plainnat.bst}
% \bibliographystyle{/usr/share/texmf/bibtex/bst/natbib/unsrtnat.bst}
% \bibliographystyle{unsrtnat}

\newcommand{\degree}{\ensuremath{^\circ}}
\newcommand{\SPURS}{Bay of Bengal}
\newcommand{\SitePos}{10\degree{N}; 83.5\degree{E}}

%%%%%%%%%%%%%%%%%%%%%%%%%%%
% beginning of the document
%%%%%%%%%%%%%%%%%%%%%%%%%%%
\pagestyle{empty}

\begin{document}

\mbox{ }

\vfill
\begin{center}
{\Large\textbf{Bay of Bengal Studies: Pre-cruise
environmental report}}

\vfill

\vspace{2\baselineskip} 
Compiled by:\

Chris M.\ Duncombe Rae 

Upper Ocean Processes Group, Woods Hole Oceanographic Institution
\vspace{2\baselineskip}

\vfill
February 14, 2012
\vfill
\end{center}
\clearpage


\begin{center}
\large\textbf{Bathymetry}
\end{center}


 Contours from ETOPO-1
(1-arc-minute gridded data) are  drawn. Multi-beam data are
not available. The proposed \SPURS{} site (\SitePos) is indicated by the red B (between 3600 and 3650~m
depth). (a) 4-degree square around site;  contours are drawn at
250~m intervals. (b) 1-degree square around site; contours are
drawn at 50~m intervals and annotated at 200m intervals.
(c) GEBCO 0.8 degree gridded data.

\setlength{\unitlength}{1in}

\begin{picture}(9,5)(0,0)
% \makebox[3.25in][r]{\includegraphics[width=4.0in]{map.ps}}\raisebox{1.6cm}{\includegraphics[width=4.0in]{zoommap.ps}}
% \vspace{-0.75in} 
% \includegraphics[width=4.0in]{gebcomap.ps}
% \end{center}
\put(-1,0){\includegraphics[width=4.0in]{map.ps}}}
\put(3,0.66){\includegraphics[width=4.0in]{zoommap.ps}}
\put(1,-3.4){\includegraphics[width=4.0in]{gebcomap.ps}}

\end{picture}

\clearpage

\begin{center}
\large \textbf{Wind Speed}
\end{center}

% Data at the SPURS site were obtained from SCOW
% (\url{http://cioss.coas.oregonstate.edu/scow/citation.html})
% 
% \vspace{2\baselineskip}
% \begin{center}
% \includegraphics[width=4in]{fig_20111204T113318.432483.eps}
% \end{center}
% 
% \clearpage
% Merged Near Real Time winds were obtained from AVISO at 
% \url{http://opendap.aviso.oceanobs.com/thredds/dodsC/dataset-nrt-global-merged-mwind}
% 
% QuikSCAT Winds from AVISO are confusing me. The numbers are half
% what the SCOW data suggests, and the variable provided in the NetCDF
% files is `Modulus Wind Speed' with units `m' and not `m/s' 
 % that one might expect. Perhaps Modulus means
% something different than I think but elsewhere Modulus Wind
% Speed has units `m/s', and other typos appear in the AVISO web
% pages, so it suggests the units are really `m/s', but why are the
% numbers half the other source? 
% 
% Here are (a) wind speed time series (mean of data points in
% 2-degree square around
% SPURS site) (b) Seasonal wind speeds (all data points in 2-degree
% square around SPURS site) (c) all data in 2-degree square around
% SPURS site. 
% 
% UPDATE: Further on the variable plotted here: It is clearly indicated on the
% download page that these data are `Modulus Wind Speed' with units `m'.
% Perhaps they mean SWH computed from wind speed, but that is not what they
% say on the download page, nor in the header to the NetCDF file. 
% 
% \begin{center}
% \includegraphics[width=4in]{aviso_winds_1.eps}
% 
% \includegraphics[width=3.2in]{aviso_winds_2.eps}
% \includegraphics[width=3.2in]{aviso_winds_3.eps}
% \end{center}
% 
% 
% 
% 
% \clearpage
QuikSCAT winds were obtained from Remote Sensing Systems (RSS) at  
\url{http://www.ssmi.com/data/qscat/bmaps_v04/y2009/m11/}
% These quikscat winds are about same magnitude as SCOW. 

\begin{center}
\includegraphics[width=4in]{ssmi_fig_1.eps}

\includegraphics[width=3.2in]{ssmi_fig_2.eps}
\includegraphics[width=3.2in]{ssmi_fig_3.eps}
\end{center}

\clearpage

\begin{center}
\large\textbf{Wave Height}
\end{center}

Significant wave height data from AVISO were found at
\url{http://www.aviso.oceanobs.com/en/data/products/wind-waves-products/index.html}
and were downloaded from
\url{ftp://ftp.aviso.oceanobs.com/pub/oceano/AVISO/wind-wave/nrt/mswh/merged/}. 

The waves data were extracted for the 1x1-degree square at the
\SPURS{} position
(\SitePos) and are shown in the Figures.

\vspace{2\baselineskip}
\includegraphics[width=3in]{waves_fig_1.eps}
\raisebox{0.1in}{\includegraphics[width=3in]{waves_fig_2.eps}}

\clearpage


\begin{center}
\large\textbf{Flux}
\end{center}

Product from the WHOI Objectively Analyzed Air-Sea Fluxes (OAFlux) Project was
obtained \url{http://oaflux.whoi.edu} as NetCDF files consisting of monthly
means of the following variables:
 
\begin{itemize}
\item Evaporation
\item Latent Heat Flux
\item Surface Sensible Heat Flux
\item Specific Humidity at 2m
\item Neutral Wind Speed at 10m
\item Sea Surface Temperature
\item Air Temperature at 2m
\item Net Surface Heat Flux
\item Net Surface Longwave Radiation Flux
\item Net Surface Shortwave Radiation Flux
\end{itemize}

The first seven variables cover an eleven year period from 2000
to 2010, the last three (qnet, lwr and swr) are over a twelve
year period from 1996 to 2007.

These are shown below in three figures, as a time series, a monthly mean,
and a boxplot demonstrating ranges and location. The mean is plotted as a
green diamond on the boxplot.  The red line indicates the median.

% I applied all obvious scaling from the NetCDF
% files, and the temperatures and wind values look plausible, but I am not
% familiar enough with the expected ranges of the other variables to comment
% about them. 


\includegraphics[width=2in]{oafluxfig01.eps}
\includegraphics[width=2in]{oafluxfig02.eps}
\includegraphics[width=2in]{oafluxfig03.eps}

\includegraphics[width=2in]{oafluxfig04.eps}
\includegraphics[width=2in]{oafluxfig05.eps}
\includegraphics[width=2in]{oafluxfig06.eps}

\includegraphics[width=2in]{oafluxfig07.eps}
\includegraphics[width=2in]{oafluxfig08.eps}
\includegraphics[width=2in]{oafluxfig09.eps}

\includegraphics[width=2in]{oafluxfig10.eps}
\includegraphics[width=2in]{oafluxfig11.eps}
\includegraphics[width=2in]{oafluxfig12.eps}

\includegraphics[width=2in]{oafluxfig13.eps}
\includegraphics[width=2in]{oafluxfig14.eps}
\includegraphics[width=2in]{oafluxfig15.eps}

\includegraphics[width=2in]{oafluxfig16.eps}
\includegraphics[width=2in]{oafluxfig17.eps}
\includegraphics[width=2in]{oafluxfig18.eps}

\includegraphics[width=2in]{oafluxfig19.eps}
\includegraphics[width=2in]{oafluxfig20.eps}
\includegraphics[width=2in]{oafluxfig21.eps}

\includegraphics[width=2in]{oafluxfig22.eps}
\includegraphics[width=2in]{oafluxfig23.eps}
\includegraphics[width=2in]{oafluxfig24.eps}

\includegraphics[width=2in]{oafluxfig25.eps}
\includegraphics[width=2in]{oafluxfig26.eps}
\includegraphics[width=2in]{oafluxfig27.eps}

\includegraphics[width=2in]{oafluxfig28.eps}
\includegraphics[width=2in]{oafluxfig29.eps}
\includegraphics[width=2in]{oafluxfig30.eps}


% \bibliography{./abs,./cmdrrefs}

\clearpage

\begin{center}
\large\textbf{Precipitation}
\end{center}

Precipitation data were retrieved from\
\url{http://www1.ncdc.noaa.gov/pub/data/gpcp/1dd-v1.1/} (directed thereto
from \url{http://www.gewex.org/gpcpdata.htm}).  Data are for 1-degree
blocks centered on the half-degree.  Daily precipitation data were
extracted at the {\SPURS} site (\SitePos) by taking the mean of the
surrounding four data points, and then manipulating to obtain the monthly
descriptors. Shown are a time series and monthly mean presentations
indicating the seasonal variability.

\begin{center}
\includegraphics[width=3in]{gpcp_fig_1.eps}
\includegraphics[width=3in]{gpcp_fig_3.eps} 

\includegraphics[width=3in]{gpcp_fig_2.eps}
\includegraphics[width=3in]{gpcp_fig_4.eps}

\includegraphics[width=3in]{gpcp_fig_5.eps}
\end{center}

\clearpage

\begin{center}
\large\textbf{Near-surface Hydrography}
\end{center}

ARGO data for the two-degree square around \SPURS{} site (\SitePos) were
obtained from NODC and plotted. Shown are data sites and sample depths;
vertical temperature, salinity and density; and temperature vs salinity.  Color
indexed scatter plots of the Temperature and Salinity show vertical T and S
distribution. 

\begin{center}
\includegraphics[width=3in]{argo_1_2deg.eps}
\includegraphics[width=3in]{argo_6_2deg.eps}

\includegraphics[width=3in]{argo_3_2deg.eps} 
\includegraphics[width=3in]{argo_4_2deg.eps}

\includegraphics[width=3in]{argo_2_2deg.eps}
\includegraphics[width=3in]{argo_5_2deg.eps} 

\includegraphics[width=3in]{argo_7_2deg.eps}
\includegraphics[width=3in]{argo_8_2deg.eps} 
\end{center}

\clearpage

As above for ARGO data within 5-degree square around \SPURS{} site. Otherwise
as for the 2-degree box around \SPURS{} site.

\begin{center}
\includegraphics[width=3in]{argo_1_5deg.eps}
\includegraphics[width=3in]{argo_6_5deg.eps}

\includegraphics[width=3in]{argo_3_5deg.eps} 
\includegraphics[width=3in]{argo_4_5deg.eps}

\includegraphics[width=3in]{argo_2_5deg.eps}
\includegraphics[width=3in]{argo_5_5deg.eps} 

\includegraphics[width=3in]{argo_7_5deg.eps}
\includegraphics[width=3in]{argo_8_5deg.eps} 
\end{center}


\end{document}

% vi: se nowrap tw=0 :

